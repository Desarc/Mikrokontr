\documentclass[a4paper,11pt]{article}

\usepackage[plain]{fullpage} %Error.
\usepackage{graphicx}  %This enables the inclusion of pdf graphic files in figures
\usepackage{wrapfig}
\usepackage{array}
\usepackage[hidelinks]{hyperref}
\usepackage{color}
\definecolor{light-gray}{gray}{0.95}
\usepackage{listings}
\lstset{
basicstyle=\footnotesize, 
morecomment=[l]{/*},
backgroundcolor=\color{light-gray}, 
xleftmargin=.10in,
xrightmargin=.10in,
}

\title{\textbf{Report: Exercise 2}}
\author{Group 9: \O yvin Richardsen, Sandor Zeestraten, Stian Habbestad}
\date{{Norwegian University of Science and Technology \\
TDT4258 Energy Efficient Computer Design \\}
\today}
 
 
 
\begin{document}
\maketitle

\begin{abstract}

\end{abstract}

\tableofcontents
\newpage

\section{Introduction} 



\section{Description and methodology}
We started out with implementing the previous exercise that was coded in assembly, in C. This was an easy start for implementing an interrupt routine, as well as the basics we needed for this exercise. Next step was to enable and setup the ABDAC. First out we started to generate noise, just to easily hear if it worked. 

\section{Results}

\section{Tests}
\paragraph{Description}


\paragraph{Results}
\addcontentsline{toc}{subsection}{Results}
Below is a table of the different tests we ran, the preconditions and the results.
\linebreak

\renewcommand{\arraystretch}{1.25} %vertical cell padding
\begin{tabular}[pos]{|m{70pt}|m{90pt}|m{90pt}|m{100pt}|m{60pt}|}
\hline
\textbf{Name} 				& \textbf{Preconditions}				& \textbf{Description} 					& \textbf{Expected result} 													& \textbf{Test result} 		\\ \hline

Steady-state test			& Power is on, and the board is connected & Upload program to card, push reset switch 	& The board is powered and LED 7 should be on 									& Passed 				\\ \hline

Move LED right test			& Program is active, only LED 7 is on 		& Push switch 6						 	& LED 7 should be turned off and LED 6 should be turned on 							& Passed 				\\ \hline

Move LED left test			& Program is active, only LED 6 is on 		& Push switch 7						 	& LED 6 should be turned off and LED 7 should be turned on 							& Passed 				\\ \hline

Left LED wrap-around test		& Program is active, only LED 7 is on 		& Push switch 67					 	& LED 7 should be turned off and LED 0 should be turned on 							& Passed 				\\ \hline

Right LED wrap-around test	& Program is active, only LED 0 is on 		& Push switch 6						 	& LED 0 should be turned off and LED 7 should be turned on 							& Passed 				\\ \hline

Long push test				& Program is active, only LED 7 is on 		& Push and hold switch 6 for a few seconds 	& LED 7 should be turned off and LED 6 should be turned on as soon as the switch is pushed 	& Passed 				\\ \hline
\end{tabular}

\section{Evaluation of assignment}

\section{Conclusion}

\section{Appendix}

\footnotesize{  % This makes the Reference items print in footnotesize fonts
\begin{thebibliography}{N}

\bibitem[1]{secure} TDT4258 Compendium
\url{http://www.idi.ntnu.no/emner/tdt4258/_media/kompendium.pdf}

\end{thebibliography}.  
}

\end{document} 
